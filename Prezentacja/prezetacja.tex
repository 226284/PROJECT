\documentclass{beamer}
\usepackage[utf8]{inputenc}
\usepackage[MeX]{polski}
\usepackage{graphicx}
\usepackage{graphicx}
\usepackage{listings}
\usetheme{Warsaw}
\usecolortheme{wolverine}
\setbeamerfont{title}{family=\rm}
\setbeamerfont{author}{family=\it}
\title{Problem plecakowy}
\subtitle{Projektowanie algorytmów i metod sztucznej inteligencji}
\author{Michał Wieczorek, Artur Szafraniak}
\institute{
Automatyka i Robotyka,
Wydział Elektroniki\\
Politechnika Wrocławska}

\begin{document}
\begin{frame}
\titlepage
\end{frame}

\section{Spis treści}
\begin{frame}
	\frametitle{Plan prezentacji}
	\tableofcontents
\end{frame}

\section{Wprowadzenie}
\subsection{Na czym polega ten problem}
\begin{frame}
	\frametitle{Na czym polega problem plecakowy}
	\begin{figure}[H]
	\centering
	\includegraphics[scale=0.3]{zlodziej2.png}
	\end{figure}
\end{frame}

\section{Sposoby rozwiązania}
\subsection{Algorytmy zachłanne}
\begin{frame}
	\frametitle{Rodzaje algorytmów zachłannych}
	\begin{itemize}
	\item Sortowanie według wartości towaru
	\item Sortowanie według objętości
	\item Sortowanie według współczynnika wartość/objętość
	\end{itemize}
\end{frame}

\subsection{Algorytm Knapsack 0-1}
\begin{frame}
	\frametitle{Zasada działania}
	Tu będzie opisa działania algorytmu
\end{frame}

\begin{frame}[fragile]
	\frametitle{Algorytm \textit{Knapsack 0-1}}
	\begin{lstlisting}[basicstyle=\tiny,tabsize=2]
int Magazyn::knapsack(int wielkosc) {
int i, j; // pomocnicze liczniki
int tmp[ROZMIAR + 1][wielkosc + 1]; // tablica pomocnicza do przechowywania danych

for (i = 0; i <= ROZMIAR; i++) {
	for (j = 0; j <= wielkosc; j++) {
		if (i == 0 || j == 0) { // zerowe indeksy wypelniamy zerami
			tmp[i][j] = 0;
		}
		else if (tab[i - 1].get_masa() <= j) {
			// znalezienie maksimum
			tmp[i][j] = max(
			tab[i - 1].get_wartosc() + tmp[i - 1][j - tab[i - 1].get_masa()],
			tmp[i - 1][j]);
		}
		else { // zwykle przepisanie z wyzszego indeksu tablicy
			tmp[i][j] = tmp[i - 1][j];
		}
	}
}
return tmp[ROZMIAR][wielkosc]; // zwrocenie maksymalnej wartosci
}
	\end{lstlisting}
\end{frame}

\end{document}