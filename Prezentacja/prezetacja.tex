\documentclass{beamer}
\usepackage[utf8]{inputenc}
\usepackage[MeX]{polski}
\usepackage{graphicx}
\usepackage{listings}
\usepackage{array}
\usetheme{Warsaw}
\usecolortheme[rgb={0,0.5,1}]{structure}
\setbeamerfont{title}{family=\rm}
\setbeamerfont{author}{family=\it}
\title{Problem plecakowy}
\subtitle{Projektowanie algorytmów i metod sztucznej inteligencji}
\author{Michał Wieczorek, Artur Szafraniak}
\institute{
Automatyka i Robotyka,
Wydział Elektroniki\\
Politechnika Wrocławska}

\begin{document}
\begin{frame}
\titlepage
\end{frame}

\section{Spis treści}
\begin{frame}
	\frametitle{Plan prezentacji}
	\tableofcontents
\end{frame}

\section{Wprowadzenie}
\subsection{Na czym polega ten problem}
\begin{frame}
	\frametitle{Na czym polega problem plecakowy}
	\begin{figure}[H]
	\centering
	\includegraphics[scale=0.3]{zlodziej2.png}
	\end{figure}
\end{frame}

\section{Sposoby rozwiązania}
\subsection{Algorytmy zachłanne}
\begin{frame}
	\frametitle{Rodzaje algorytmów zachłannych}
	\begin{itemize}
	\item Sortowanie według wartości towaru
	\item Sortowanie według objętości
	\item Sortowanie według współczynnika wartość/objętość
	\end{itemize}
\end{frame}

\subsection{Algorytm Knapsack 0-1}
\begin{frame}
	\frametitle{Zasada działania}
	
\end{frame}

\begin{frame}[fragile]
	\frametitle{Algorytm \textit{Knapsack 0-1}}
	\begin{lstlisting}[basicstyle=\tiny,tabsize=2]
void Magazyn::knapsack(int wielkosc) {
	int i, j; // pomocnicze liczniki
	int tmp[ROZMIAR + 1][wielkosc + 1]; // tablica pomocnicza do przechowywania danych

	for (i = 0; i <= ROZMIAR; i++) {
		for (j = 0; j <= wielkosc; j++) {
			if (i == 0 || j == 0) { // zerowe indeksy wypelniamy zerami
				tmp[i][j] = 0;
			}
			else if (tab[i].get_masa() <= j) {
				// znalezienie maksimum
				tmp[i][j] = max(
						tab[i].get_wartosc()
								+ tmp[i - 1][j - tab[i].get_masa()],
						tmp[i - 1][j]);
			}
			else { // zwykle przepisanie z wyzszego indeksu tablicy
				tmp[i][j] = tmp[i - 1][j];
			}
		}
	}
	\end{lstlisting}
\end{frame}

\subsection{Wybieranie elementów do plecaka}
\begin{frame}[fragile]
\frametitle{Wybieranie elementów do plecaka}
\begin{lstlisting}[basicstyle=\small, tabsize=2]
	i = ROZMIAR;
	j = wielkosc;

	while (i > 0 && j > 0) {
		if (tmp[i][j] != tmp[i - 1][j]) {
			plecak.push_back(tab[i]);
			j = j - tab[i].get_masa();
			i = i - 1;
		}
		else {
			i = i - 1;
		}
	}
	
\end{lstlisting}
\end{frame}
\begin{frame}
\begin{table}[]
\begin{tabular}{|c|c|c|}
\hline
Nazwa        & Waga & Wartość	\\ \hline
Kolczyki     & 2    & 5 	\\ \hline
Pierościonek & 3    & 7 	\\ \hline
Naszyjnik    & 4    & 9 	\\ \hline
Zegarek      & 5    & 8 	\\ \hline
\end{tabular}
\end{table}
\begin{table}[]
\begin{tabular}{|c|c|c|c|c|c|c|}
\hline
V{[}i,w{]} & w=0 & 1 & 2 & 3 & 4 & 5 \\ \hline
i=0        & \onslide<2->{0} & \onslide<2->{0} & \onslide<2->{0} &\onslide<2->{0} & \onslide<2->{0} & \onslide<2->{0}   \\ \hline
1          & \onslide<2->{0} &   &   &   &   &   \\ \hline
2         & \onslide<2->{0} &   &   &   &   &   \\ \hline
3          & \onslide<2->{0} &   &   &   &   &   \\ \hline
4          & \onslide<2->{0} &   &   &   &   &   \\ \hline
\end{tabular}
\end{table}
\end{frame}
\end{document}